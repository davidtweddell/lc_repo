\documentclass[border=40pt,tikz]{standalone}
\usepackage{graphicx}

% to use tikz
\usepackage{tikz}
\usetikzlibrary{mindmap, positioning, shapes.geometric, calc}
\usetikzlibrary{shapes.multipart, shapes.arrows, arrows.meta}
\usetikzlibrary{shadows, fit, decorations.markings}
\usetikzlibrary{backgrounds}

% \usepackage{smartdiagram}

% to use custome fonts
\usepackage[no-math]{fontspec} 
% \setmainfont[BoldFont = {Roboto Bold}]{Roboto Bold}
% \setmainfont{Roboto-Regular}[BoldFont = Roboto-Bold]
\setmainfont{Aptos}[BoldFont = Aptos-Bold]
\setmonofont{IBMPlexMono-Regular}[BoldFont = IBMPlexMono-Bold]

% to manipulate colors
\usepackage{xcolor}
% \usepackage{pgfgantt}
\usepackage{mycolor}

% to read datafiles
\usepackage{datatool}

% page geometry
\usepackage[left=1.50cm, right=1.50cm, top=1.50cm, bottom=2.00cm]{geometry}
% % headers and footers

%--------------------------------
%--------------------------------
\begin{document}
%--------------------------------
%--------------------------------

% %--------------------------------
% % define tikz formats
% %--------------------------------
% \tikzset{%
%         customdash/.style={dash pattern=on 2pt off 3pt on 4pt off 4pt},
%         tbcdash/.style={dash pattern=on 5pt off 5pt on 5pt off 5pt},
%         font={\fontsize{24pt}{24}\selectfont}
% }


% colours
\definecolor{orangeii}{RGB}{240,90, 10}
\definecolor{darkblue}{rgb}{0.05,0.25,0.65}


% angular position of labels, etc.
\def\startpos{90}

% radius of the circle
\def\circrad{20}

% radius for labels
\def\lblrad{\circrad + 2.0}
\def\lblindent{+0}
\def\lblsze{\small}

% # arc features
\def\arcgap{2}
\def\arcwidth{0.5mm}

\DTLloaddb{sitenodes}{./nodes-sites.csv}
\DTLloaddb{clusternodes}{./nodes-cluster.csv}
\DTLloaddb{symptomnodes}{./nodes-symptom.csv}

\DTLloaddb{clustersymptedges}{./edges-cluster-x-symptoms-ternary.csv}
\DTLloaddb{siteclusteredges}{./edges-sites-x-clusters-ternary.csv}
% \DTLloaddb{siteclusteredges}{./edges-cluster-x-site.csv}
\DTLloaddb{sitesympedges}{./edges-sites-x-symptoms-ternary.csv}


% size of nodes on joining lines
\def\noderad{.25}
    
% \input{colourdefs.tex}
\definecolor{clustercolour1}{RGB}{251.999925, 79.00002, 47.999925000000005}
\definecolor{clustercolour2}{RGB}{228.999945, 174.000015, 56.00004}
\definecolor{clustercolour3}{RGB}{109.000005, 144.00003, 79.00002}
\definecolor{clustercolour4}{RGB}{138.99999, 138.99999, 138.99999}
\definecolor{clustercolour5}{RGB}{22.99998, 189.99999, 207.00007499999998}
\definecolor{clustercolour6}{RGB}{147.99996000000002, 103.00011, 188.99988}

\definecolor{featurecolour1}{RGB}{205.999965, 133.000095, 255.0}
\definecolor{featurecolour2}{RGB}{0.0, 102.0, 119.99994}
\definecolor{featurecolour3}{RGB}{82.99995, 68.99994, 138.99999}



\begin{tikzpicture}[double distance = 3mm, regular polygon sides = 3]
    % layers
    \pgfdeclarelayer{nodes}
    \pgfdeclarelayer{edges}
    \pgfsetlayers{main,edges,nodes}

    \tikzset{
        dot/.style={circle, fill=black, inner sep=0pt, minimum size=14pt},
        chord/.style={
                        white, 
                        line width=1mm,
                        double=gray,
                        double distance=1mm, 
                        cap=round,
            % decoration={markings, mark=between positions 0 and 1 step .999 with 
            % {\node[dot]{};},},
        },
        nstyle/.style={circle, 
                        draw,
                        line width=0.5mm,
                        minimum size=4pt, 
                        inner sep=+0pt, 
                        fill=blue!10, 
                        text=black, 
                        % color = white,
                        text width= 2cm, 
                        align = center},
    }
    
% draw a grid for reference

% draw a wedge frin 0 to 120 degrees
% \draw[fill=blue!10] (0:0) -- (60:20) arc (180:20) -- cycle;

% draw a sector of a circle from 60 to 180 degrees
\draw[black, fill=red!10] (0:0) -- (60:\circrad) arc (60:180:\circrad) -- cycle;
\draw[black, fill=gray!10] (0:0) -- (-60:\circrad) arc (-60:60:\circrad) -- cycle;
\draw[black, fill=blue!10] (0:0) -- (180:\circrad) arc (180:300:\circrad) -- cycle;

% \draw[step=1cm,gray,very thin] (-10,-10) grid (10,10);

% draw the corners of an equilateral triangle
\coordinate (ctr) at (0:0);
\coordinate (C) at (0:\circrad+0);
\coordinate (F) at (120:\circrad+0);
\coordinate (S) at (240:\circrad+0);

% draw the triangle
\node[nstyle, text width = 4cm,fill = gray!30] at (0:\lblrad) {\Huge Symptom Cluster};
\node[nstyle, text width = 4cm,fill = blue!30] at (240:\lblrad) {\Huge Site};
\node[nstyle, text width = 4cm,fill = red!30] at (120:\lblrad) {\Huge Symptom};

% # TODO
% add a legend and use short labels?


\begin{pgfonlayer}{nodes}

%--------------------------------
% Clusters
%--------------------------------
\path (ctr) -- (C)
node foreach \i in {1, ..., 6}
[nstyle, fill=blue!20, pos=\i/7, name = c\i,]{Cluster \i}; 

\DTLforeach{clusternodes}{%
\LBL=LABEL,\CLR=COLOUR}{%

% get the row number
    \def\rownum{\DTLcurrentindex}
    \draw (c\rownum) node[nstyle, fill=\CLR] {\Large \rownum};
    }

% make a legend to the right of the main figure

% \node[anchor=west, text width=5cm] at (25,0) {\Large
%     \begin{tabular}{ll}
%         \textbf{Cluster} & \textbf{Colour} \\
%         \hline
%         1 & \textcolor{clustercolour1}{\rule{1cm}{1cm}} \\
%         2 & \textcolor{clustercolour2}{\rule{1cm}{1cm}} \\
%         3 & \textcolor{clustercolour3}{\rule{1cm}{1cm}} \\
%         4 & \textcolor{clustercolour4}{\rule{1cm}{1cm}} \\
%         5 & \textcolor{clustercolour5}{\rule{1cm}{1cm}} \\
%         6 & \textcolor{clustercolour6}{\rule{1cm}{1cm}} \\
%     \end{tabular}
% };
% \node[anchor=west, text width=5cm] at (25,10) {\large
%     \begin{tabular}{ll}
%         \textbf{Symptom} & \textbf{Colour} \\
%         \hline
%         \DTLforeach{symptomnodes}{%
%         \LBL=LABEL,\CLR=COLOUR}{%
%             \def\rownum{\DTLcurrentindex}
%             \LBL & \textcolor{\CLR}{\rule{1cm}{1cm}} \\
%         }

%     \end{tabular}
% };




%--------------------------------
% Sites
%--------------------------------
\path (ctr) -- (S)
node foreach \i in {1, ..., 5}
[nstyle, fill=red!20, pos=\i/6, name = site\i,]{Site \i}; 

\DTLforeach{sitenodes}{%
\SITE=SITE,\LBL=LABEL}{%

% get the row number
\def\rownum{\DTLcurrentindex}
\draw (site\rownum) node[nstyle,fill=gray, text=white, ] {\Large \LBL};
}

%--------------------------------
% Symtpoms
%--------------------------------
\path (ctr) -- (F)
    node foreach \i in {1, ..., 9}
    [nstyle, fill=gray!10, pos=\i/10, name=f\i]{Feature \i}; 

\DTLforeach{symptomnodes}{%
\LBL=LABEL,\CLR=COLOUR,\TCLR=TEXTCOLOUR}{%
    \def\rownum{\DTLcurrentindex}
    \draw (f\rownum) node[nstyle, fill=\CLR, text=\TCLR,
    % make text larger
    ] {\large \LBL};
    }

\end{pgfonlayer}{nodes}



%--------------------------------
%--------------------------------
% edges
%--------------------------------
%--------------------------------
\begin{pgfonlayer}{edges}
    \DTLforeach{clustersymptedges}{%
    \WIDTH=WIDTH,\ES=EDGESTRING,\CLR=EDGECOLOUR}{%
    \draw[chord, double distance =\WIDTH, double=\CLR] \ES;}

    \DTLforeach{siteclusteredges}{%
    \WIDTH=WIDTH,\ES=EDGESTRING,\CLR=EDGECOLOUR}{%
    \draw[chord, double distance =0.2*\WIDTH, double=\CLR] \ES;}

    \DTLforeach{sitesympedges}{%
    \WIDTH=WIDTH,\ES=EDGESTRING,\CLR=EDGECOLOUR}{%
    \draw[chord, double distance =1*\WIDTH, double=\CLR] \ES;}
\end{pgfonlayer}{edges}

\end{tikzpicture}

%--------------------------------
%--------------------------------
\end{document}
%--------------------------------
%--------------------------------